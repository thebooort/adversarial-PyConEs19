% last updated in April 2002 by Antje Endemann
% Based on CVPR 07 and LNCS, with modifications by DAF, AZ and elle, 2008 and AA, 2010, and CC, 2011; TT, 2014; AAS, 2016

\documentclass[runningheads]{llncs}
\usepackage[english,spanish.notilde.lcroman,shorthands=off]{babel}
\usepackage{graphicx}
\usepackage{amsmath,amssymb} % define this before the line numbering.
\usepackage{ruler}
\usepackage{color}
\usepackage{enumitem,amssymb}
\newlist{todolist}{itemize}{2}
\setlist[todolist]{label=$\square$}
\usepackage{pifont}
\newcommand{\cmark}{\ding{51}}%
\newcommand{\done}{\rlap{$\square$}{\raisebox{2pt}{\large\hspace{1pt}\cmark}}%
\hspace{-2.5pt}}
\usepackage[width=122mm,left=12mm,paperwidth=146mm,height=193mm,top=12mm,paperheight=217mm]{geometry}

\usepackage{fancyhdr}
\fancyhf{}
\renewcommand{\headrulewidth}{0pt}
\renewcommand{\footrulewidth}{0pt}
\setlength\headheight{80.0pt}
\addtolength{\textheight}{-80.0pt}
%\chead{\includegraphics[width=0.15\textwidth]{Largo.PNG}}

\begin{document}

\pagestyle{headings}
\mainmatter


\title{Cariño, he optimizado a los niños.} % Replace with your title


\author{Propuesta an\'onima para PyConES 2019}

\maketitle

\section{Tipo de Contribuci\'on}

\begin{todolist}
  \item Charla corta
  \item [\done]Charla extendida
  \item P\'oster
  \item Taller
  \end{todolist}


\section{Idioma}
\begin{todolist}
  \item [\done]Espa\~nol
  \item Ingl\'es
\end{todolist}
\section{Nivel}

\begin{todolist}
  \item Avanzado
  \item [\done] Medio
  \item Iniciaci\'on
  \end{todolist}

\keywords{Python, Optimización, Algoritmos evolutivos, Optimización Bayesiana, Problema del viajante}

\newpage

\section{Resumen}
\textit{Asumámoslo, todos somos bastante holgazanes. Optimizar está en nuestra naturaleza, no sólo en la humana sino en la del mundo que nos rodea. Y por consecuencia, también en nuestra tecnología. Distintos procesos son el producto de una optimización como la forma de las pompas de jabón, la ruta que ha de seguir un repartidor o la inversión que se ha de realizar en bolsa. }
	
	\textit{Pero conforme nuestro mundo y nuestras vidas evolucionan también lo hacen los problemas a los que nos enfrentamos y los recursos necesarios para sobrepasarlos con éxito. La eficiencia es la clave para la mayoría de las aplicaciones presentes y futuras, pero llegar a conseguirla no será fácil. La optimización proporciona una combinación elegante entre la teoría matemática más pura y su aplicación en el mundo que nos rodea.}
	
	\textit{En esta charla queremos motivar y explicar en que consisten estos problemas de optimización, que dificultades podemos encontrar y como podremos resolverlas usando Python. Para ello usaremos distintos enfoques desde el más simple que consiste en derivar una función, métodos numéricos como gradiente descendiente, algoritmos bioinspirados como los evolutivos y por último optimización bayesiana que tiene en cuenta la incertidumbre y en pocas iteraciones puede encontrar el óptimo.}
	
	\textit{Finalmente, pondremos el famoso ejemplo de optimización conocido como el problema del viajante para resolverlo entre todos con los métodos expuestos.}

\section{Presentaci\'on}
..


\section{Pre-requisitos para atender a la presentación}
\begin{itemize}
\item  Los asistentes deber\'an tener conocimientos previos de conceptos básicos de matemáticas. 
\end{itemize}

\section{Otros requerimientos t\'ecnicos}
\begin{itemize}
 \item Un puerto VGA/HDMI para poder enchufar nuestro portatil.
\end{itemize}

\clearpage

\bibliographystyle{splncs}
\bibliography{egbib}
\end{document}
