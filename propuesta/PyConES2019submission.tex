% last updated in April 2002 by Antje Endemann
% Based on CVPR 07 and LNCS, with modifications by DAF, AZ and elle, 2008 and AA, 2010, and CC, 2011; TT, 2014; AAS, 2016

\documentclass[runningheads]{llncs}
\usepackage[english,spanish.notilde.lcroman,shorthands=off]{babel}
\usepackage{graphicx}
\usepackage{amsmath,amssymb} % define this before the line numbering.
\usepackage{ruler}
\usepackage{color}
\usepackage{enumitem,amssymb}
\newlist{todolist}{itemize}{2}
\setlist[todolist]{label=$\square$}
\usepackage{pifont}
\newcommand{\cmark}{\ding{51}}%
\newcommand{\done}{\rlap{$\square$}{\raisebox{2pt}{\large\hspace{1pt}\cmark}}%
\hspace{-2.5pt}}
\usepackage[width=122mm,left=12mm,paperwidth=146mm,height=193mm,top=12mm,paperheight=217mm]{geometry}

\usepackage{fancyhdr}
\fancyhf{}
\renewcommand{\headrulewidth}{0pt}
\renewcommand{\footrulewidth}{0pt}
\setlength\headheight{80.0pt}
\addtolength{\textheight}{-80.0pt}
%\chead{\includegraphics[width=0.15\textwidth]{Largo.PNG}}

\begin{document}

\pagestyle{headings}
\mainmatter


\title{Cariño, he optimizado a los niños.} % Replace with your title


\author{Propuesta an\'onima para PyConES 2019}

\maketitle

\section{Tipo de Contribuci\'on}

\begin{todolist}
  \item Charla corta
  \item [\done]Charla extendida
  \item P\'oster
  \item Taller
  \end{todolist}


\section{Idioma}
\begin{todolist}
  \item [\done]Espa\~nol
  \item Ingl\'es
\end{todolist}
\section{Nivel}

\begin{todolist}
  \item Avanzado
  \item [\done] Medio
  \item Iniciaci\'on
  \end{todolist}

\keywords{Python, Optimización, Algoritmos evolutivos, Optimización Bayesiana, Problema del viajante}

\newpage

\section{Resumen}
\textit{La optimización matemática se puede definir como encontrar el valor que maximiza o minimiza una determinada función. Muchos problemas que surgen de la física, ingeniería, economía, ... se pueden reformular como la búsqueda de un punto óptimo de una función: encontrar el punto que minimiza la energía, que minimiza costes, que minimiza recursos, ... 
	
En esta charla queremos motivar y explicar en que consisten estos problemas de optimización, que dificultades podemos encontrar y como podremos resolverlas usando Python. Para ello usaremos distintos enfoques desde el más simple que consiste en derivar una función, métodos numéricos como gradiente descendiente, algoritmos bioinspirados como los evolutivos y por último optimización bayesiana que tiene en cuenta la incertidumbre y en pocas iteraciones puede encontrar el óptimo.

Finalmente, pondremos el famoso ejemplo de optimización conocido como el problema del viajante para resolverlo entre todos con los métodos expuestos.}

\section{Presentaci\'on}

\textit{
	
	Aunque no existe un l\'imite estricto de palabras, una extensi\'on normal no suele ser superior a una p\'agina. Esta presentaci\'on puede contener fotograf\'ias, gr\'aficos, y enlaces a demos y/o c\'odigo (el cual no se incluir\'a en la propuesta). En la propuesta debe distinguirse de manera clara los aspectos m\'as relevantes para que la propuesta sea inteligible (ejemplo: Introducci\'on, metodolog\'ia empleada, resultados que se han obtenido, demos y/o fragmentos de c\'odigo imprescindibles \dots). Adem\'as, se deber\'an aportar las conclusiones m\'as destacadas de la propuesta y las referencias consultadas.}
\cite{Alpher02},
\cite{Alpher03}, \cite{Alpher04} \dots


\section{Pre-requisitos para atender a la presentación}
\begin{itemize}
\item  Los asistentes deber\'an tener conocimientos previos de conceptos básicos de matemáticas. 
\end{itemize}

\section{Otros requerimientos t\'ecnicos}
\begin{itemize}
 \item Un puerto VGA/HDMI para poder enchufar nuestro portatil.
\end{itemize}

\clearpage

\bibliographystyle{splncs}
\bibliography{egbib}
\end{document}
