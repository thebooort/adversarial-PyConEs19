% last updated in April 2002 by Antje Endemann
% Based on CVPR 07 and LNCS, with modifications by DAF, AZ and elle, 2008 and AA, 2010, and CC, 2011; TT, 2014; AAS, 2016

\documentclass[runningheads]{llncs}
\usepackage[english,spanish.notilde.lcroman,shorthands=off]{babel}
\usepackage{graphicx}
\usepackage{amsmath,amssymb} % define this before the line numbering.
\usepackage{ruler}
\usepackage{color}
\usepackage{enumitem,amssymb}
\newlist{todolist}{itemize}{2}
\setlist[todolist]{label=$\square$}
\usepackage{pifont}
\newcommand{\cmark}{\ding{51}}%
\newcommand{\done}{\rlap{$\square$}{\raisebox{2pt}{\large\hspace{1pt}\cmark}}%
\hspace{-2.5pt}}
\usepackage[width=122mm,left=12mm,paperwidth=146mm,height=193mm,top=12mm,paperheight=217mm]{geometry}

\usepackage{fancyhdr}
\fancyhf{}
\renewcommand{\headrulewidth}{0pt}
\renewcommand{\footrulewidth}{0pt}
\setlength\headheight{80.0pt}
\addtolength{\textheight}{-80.0pt}
%\chead{\includegraphics[width=0.15\textwidth]{Largo.PNG}}

\begin{document}

\pagestyle{headings}
\mainmatter


\title{Plantilla de \LaTeX para CFP} % Replace with your title


\author{Propuesta an\'onima para PyConES 2019}

\maketitle

\section{Tipo de Contribuci\'on}

\begin{todolist}
  \item Charla corta
  \item [\done]Charla extendida
  \item P\'oster
  \item Taller
  \end{todolist}


\section{Idioma}
\begin{todolist}
  \item [\done]Espa\~nol
  \item Ingl\'es
\end{todolist}
\section{Nivel}

\begin{todolist}
  \item Avanzado
  \item [\done] Medio
  \item Iniciaci\'on
  \end{todolist}

\keywords{Python, Image Processing, Machine learning, Neural Networks, Adversarial Attacks}

\newpage

\section{Resumen}
\textit{Cada d\'ia m\'as y m\'as dispositivos hacen uso de sistemas de inteligencia artificial para intentar mejorar nuestra vida. Casos como los coches autonomos o los sistemas de reconocimiento facial son cada día más comunes. Sin embargo, este auge ha llevado a cient\'ificos de todo el mundo a preguntarse si estos sistemas son realmente seguros y podemos descansar en ellos tareas que podrían ser cruciales. Para contestar a esta pregunta, en esta charla se presentaran los principales trabajos relacionados con las redes neuronales antag\'onicas, es decir, ataques que buscan confundir sistemas de inteligencia artifical. A trav\'es de ejemplos fundamentados en el reconocimiento de imagen, explicaremos en qu\'e consisten este tipo de vulnerabilidades, las herramientas necesarias en python para realizar nuestros propios testeos y mostraremos demos en directo sobre los ejemplos tratados.}

\section{Presentaci\'on}

\textit{Aunque no existe un l\'imite estricto de palabras, una extensi\'on normal no suele ser superior a una p\'agina. Esta presentaci\'on puede contener fotograf\'ias, gr\'aficos, y enlaces a demos y/o c\'odigo (el cual no se incluir\'a en la propuesta). En la propuesta debe distinguirse de manera clara los aspectos m\'as relevantes para que la propuesta sea inteligible (ejemplo: Introducci\'on, metodolog\'ia empleada, resultados que se han obtenido, demos y/o fragmentos de c\'odigo imprescindibles \dots). Adem\'as, se deber\'an aportar las conclusiones m\'as destacadas de la propuesta y las referencias consultadas.}
\cite{Alpher02},
\cite{Alpher03}, \cite{Alpher04} \dots


\section{Pre-requisitos para atender a la presentación}
\begin{itemize}
\item  Los asistentes deber\'an tener conocimientos previos de conceptos básicos de aprendizaje autom\'atico y apredizaje profundo. 
\end{itemize}

\section{Otros requerimientos t\'ecnicos}
\begin{itemize}
 \item Durante la presentación haremos algunas demos sobre sistemas de inteligencia artificial hackeados, como vamos a llevar nuestro ordenador, nos gustaría tener un par de minutos previos a la charla para acoplar nuestro portatil.

 \item Utilizaremos nuestro ordenador  para la propuesta con lo que nos gustaría estar seguros de que habrá puerto HDMI o VGA disponible. 
\end{itemize}

\clearpage

\bibliographystyle{splncs}
\bibliography{egbib}
\end{document}
