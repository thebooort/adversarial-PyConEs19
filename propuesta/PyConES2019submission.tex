% last updated in April 2002 by Antje Endemann
% Based on CVPR 07 and LNCS, with modifications by DAF, AZ and elle, 2008 and AA, 2010, and CC, 2011; TT, 2014; AAS, 2016

\documentclass[runningheads]{llncs}
\usepackage[utf8]{inputenc}
\usepackage[spanish]{babel}
\usepackage{graphicx}
\usepackage{amsmath,amssymb} % define this before the line numbering.
\usepackage{ruler}
\usepackage{color}
\usepackage{enumitem,amssymb}
\newlist{todolist}{itemize}{2}
\setlist[todolist]{label=$\square$}
\usepackage{pifont}
\newcommand{\cmark}{\ding{51}}%
\newcommand{\done}{\rlap{$\square$}{\raisebox{2pt}{\large\hspace{1pt}\cmark}}%
\hspace{-2.5pt}}
\usepackage[width=122mm,left=12mm,paperwidth=146mm,height=193mm,top=12mm,paperheight=217mm]{geometry}

\usepackage{fancyhdr}
\fancyhf{}
\renewcommand{\headrulewidth}{0pt}
\renewcommand{\footrulewidth}{0pt}
\setlength\headheight{80.0pt}
\addtolength{\textheight}{-80.0pt}
%\chead{\includegraphics[width=0.15\textwidth]{Largo.PNG}}

\begin{document}

\pagestyle{headings}
\mainmatter


\title{Cariño, he optimizado a los niños.} % Replace with your title


\author{Propuesta an\'onima para PyConES 2019}

\maketitle

\section{Tipo de Contribuci\'on}

\begin{todolist}
  \item Charla corta
  \item [\done]Charla extendida
  \item P\'oster
  \item Taller
  \end{todolist}


\section{Idioma}
\begin{todolist}
  \item [\done]Espa\~nol
  \item Ingl\'es
\end{todolist}
\section{Nivel}

\begin{todolist}
  \item Avanzado
  \item [\done] Medio
  \item Iniciaci\'on
  \end{todolist}

\keywords{Python, Optimización, Algoritmos evolutivos, Optimización Bayesiana, Problema del viajante}

\newpage

\section{Resumen}

	
	\textit{ Asumámoslo, somos bastante holgazanes y siempre tendemos a la ley del mínimo esfuerzo. Y la tecnología no podía ser diferente. Tanto es así que muchos científicos alrededor del mundo trabajan en Optimización. La optimización matemática se puede definir como encontrar el valor que maximiza o minimiza una determinada función y por tanto responde a una definición de eficiencia. Muchos problemas que surgen de la física, ingeniería, economía o biología, se pueden reformular como la búsqueda de un punto óptimo: aquel que minimice la energía, costes o incluso recursos necesarios en una tarea. La optimización proporciona una combinación elegante entre la teoría matemática más pura y su aplicación en el mundo que nos rodea. En esta charla queremos motivar y explicar en que consisten estos problemas de optimización, qué dificultades podemos encontrar y cómo podremos resolverlas usando Python. Para ello usaremos distintos enfoques desde el más simple que consiste en derivar una función pasando por métodos numéricos como gradiente descendiente, algoritmos bioinspirados como los evolutivos hasta llegar a la optimización bayesiana, que tiene en cuenta la incertidumbre y en pocas iteraciones puede encontrar el óptimo. Finalmente, pondremos el famoso ejemplo de optimización conocido como el problema del viajante para resolverlo entre todos con los métodos expuestos.}

\section{Presentaci\'on}
\subsection{Introducción}
La optimización matemática a menudo también se llama programación no lineal, programación matemática u optimización numérica. En general podemos describirla como la ciencia del desarrollo de las  las mejores soluciones a problemas definidos matemáticamente definidos \cite{snyman2005practical}.

La definición es esencialmente abierta, pues la \textquotedblleft mejor solución\textquotedblright es un término que dependerá en exclusiva de nuestro problema. Incluso con un cuerpo teórico con años de desarrollo, estamos a merced de la complejidad inherente a la realidad. Por tanto, entender mejor la formulación matemática, y el paso de un problema real a código, nos dará una gran ventaja a la hora de enfrentarnos a este tipo de problemas.

\subsection{Desarrollo de la charla}
Motivaremos el problema de optimización con distintos ejemplos que se da en el mundo que nos rodea. A través de casos sencillos, iremos explicando distintos métodos entendiendo la dificultad creciente de los casos que se irán presentando. 

\subsubsection{Primeros casos}
El primer escollo a superar será una función 1-Dimensional. Si esta función es derivable y conocemos su derivada es bien sabido que su óptimo se encontrará en un punto crítico, es decir, un punto en el cual se anule la derivada. En este apartado, se presentarán también los principales elementos que se utilizarán en adelante \cite{bonnans2006numerical} para definir correctamente los problemas de optimización a fin de que el asistente le pierda el miedo a ver fórmulas y aprenda a comprender la idea detrás de las mismas.

\subsubsection{Gradiente Descendente}
Los primeros casos son sencillos, pero también alejados de la realidad. Muchas veces, por ejemplo es difícil encontrar los ceros de la función derivada. Para solventar los primeros problemas técnicos a los que nos enfrentamos, surgen algoritmos numéricos como el gradiente descendente \cite{ruder2016overview}, el cual de forma iterativa y, usando información de la derivada, convergerá numéricamente al punto óptimo si se cumplen las condiciones adecuadas. Con el auge actual de este tipo de métodos en problemas de aprendizaje automático, haremos una breve reseña de la importancia de entender bien cuando usamos estos algoritmos y si lo hacemos correctamente. Usaremos algunos ejemplos con \cite{scikit-learn} para ilustrar su uso.

\subsubsection{Optimización Bayesiana}
A su vez, quizás evaluar la función objetivo puede ser difícil y complicado. Pero no todo está perdido, aquí comienza la optimización bayesiana \cite{lizotte2008practical}. Este tipo de optimizacion (en ocasiones olvidada por la poca base matemática que se recibe en los estudios, pero con grandes capacidades \cite{snoek2012practical}) consigue encontrar puntos óptimos en pocas iteraciones. Además, muchas veces las observaciones de esta función objetivo son ruidosas, y querremos modelar la incertidumbre, algo que será viable con este método. Para esta parte haremos uso principalmente de \cite{gpyopt2016}.

\subsubsection{Algoritmos Bioinspirados}
En otras ocasiones, no tenemos ni información de la derivada ni nuestra función es continua (ya que puede ser discreta). Llegados a este punto debemos recurrir a metaheurísticas. Entre estos algoritmos, los más famosos son los bioinspirados, aquellos que intentan mimetizar el comportamiento de la naturaleza para alcanzar la solución. Entre estos se encuentran los algoritmos evolutivos, en los que haremos especial hincapié. Con excelentes textos como \cite{brownlee2011clever}, se pueden ofrecer ideas y conceptos clave para entender estos algoritmos, que irán acompañados de usos prácticos gracias a \cite{aarongarrett-inspyred}.
\subsubsection{Competición final}
Armados con multitud de herramientas para enfrentarnos a la dificil tarea de la optimización, propondremos el problema del viajero, explicando su objetivo, sus variantes y, finalmente, cómo puede ser resuelto usando distintos algoritmos de tipo evolutivo. Para concluir, ejecutaremos varios algoritmos y los veremos competir en una emocionante carrera hacia la solución.

\subsection{Código Abierto}

Hemos trabajado con distintos algoritmos a lo largo de nuestra carrera investigadora, por lo tanto nuestro objetivo es recopilar aquellos que nos han sido de utilidad en un repositorio de GitHub y Gitlab público.

El objetivo de esto es doble, por una parte para compartirlo con la comunidad y que las personas pueden acceder al mismo siempre que lo necesiten, y, por otra, que los asistentes puedan experimentar con los modelos propuestos y ampliar información de los mismos.

\section{Pre-requisitos para atender a la presentación}
\begin{itemize}
\item  Los asistentes deber\'an tener conocimientos previos de conceptos básicos de matemáticas. 
\item Los asistentes deber\'an tener conocimientos previos de conceptos básicos de programación. 
\item Aunque el objetivo de la charla es introducir conceptos de optimización y su implementación a través de Python, y por tanto, es una charla introductoria, hemos optado por calificarla como intermedia por algunos de los conceptos matemáticos que trataremos.
\end{itemize}

\section{Otros requerimientos t\'ecnicos}
\begin{itemize}
 \item Un puerto VGA/HDMI para poder enchufar nuestro portatil.
\end{itemize}

\clearpage

\bibliographystyle{splncs}
\bibliography{egbib}
\end{document}
